\pagenumbering{roman}
\setcounter{page}{1}

\selecthungarian

%----------------------------------------------------------------------------
% Abstract in Hungarian
%----------------------------------------------------------------------------
\chapter*{Kivonat}\addcontentsline{toc}{chapter}{Kivonat}

Különböző hálózatok, legyen az egy szociális média felhasználói hálója, egy ország közúthálózata, de akár egy NYÁK-terven a különböző komponensek vezetékezése során alapvető problémák az úgynevezett útkeresési problémák (Vehicle Routing Problems - VRP). Ezek bizonyítottan NP-nehéz problémák, a megoldásuk egy költséges, ugyanakkor gyakran monoton tevékenység, hiszen sok pontsorozat hossza közül kell kiválasztani a legrövidebbet. Dolgozatom célja a Hangyakolónia Optimalizáció (Ant Colony Optimization - ACO) elvével egy heurisztikus, vagyis valószínűségi megoldás találása. Az ACO nagymértékben párhuzamosítható, melyet azzal tudtam kihasználni, hogy videokártyán összehangolt threadek ezreit voltam képes futtatni az NVIDIA CUDA keretrendszer segítségével. A készült kódot C/C++ nyelven valósítottam meg.

%TODO Abstractot megírni

\vfill
\selectenglish


%----------------------------------------------------------------------------
% Abstract in English
%----------------------------------------------------------------------------
\chapter*{Abstract}\addcontentsline{toc}{chapter}{Abstract}

This document is a \LaTeX-based skeleton for BSc/MSc~theses of students at the Electrical Engineering and Informatics Faculty, Budapest University of Technology and Economics. The usage of this skeleton is optional. It has been tested with the \emph{TeXLive} \TeX~implementation, and it requires the PDF-\LaTeX~compiler.


\vfill
\selectthesislanguage

\newcounter{romanPage}
\setcounter{romanPage}{\value{page}}
\stepcounter{romanPage}