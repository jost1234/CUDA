\pagenumbering{roman}
\setcounter{page}{1}

\selecthungarian

%----------------------------------------------------------------------------
% Abstract in Hungarian
%----------------------------------------------------------------------------
\chapter*{Kivonat}\addcontentsline{toc}{chapter}{Kivonat}

Útkeresési problémákkal (Vehicle Routing Problems - VRP) az élet számos különböző területén találkozunk: egy szociális média felhasználói bázisának elemzése, egy NYÁK-terv elkészítése, vagy különböző logisztikai feladatok kezelése során. Ezek köztudottan NP-teljes problémák, melyek megoldása költséges, ugyanakkor gyakran monoton tevékenység, hiszen sok pontsorozat közül kell kiválasztani a legrövidebbet. Jelen dolgozat célja a Hangyakolónia Optimalizáció (Ant Colony Optimization - ACO) elvével heurisztikus, valószínűségi megoldást találni. Az ACO nagymértékben párhuzamosítható, amit azzal tudtam kihasználni, videokártyán összehangolt munkaszálak ezreit voltam képes futtatni az NVIDIA CUDA keretrendszerének segítségével. A készült kódot C/C++ programozási nyelven valósítottam meg.


Járműútvonal-tervezési problémákkal (Vehicle Routing Problems - VRP) az élet számos különböző területén találkozunk: szociális média felületek felhasználói bázisának elemzése, NYÁK-tervek készítése, vagy különböző logisztikai feladatok kezelése során. Ezek köztudottan NP-teljes problémák, melyek megoldása költséges, ugyanakkor gyakran monoton tevékenység, hiszen sok pontsorozat közül kell kiválasztani a legrövidebbet. Jelen dolgozat célja a Hangyakolónia Optimalizáció (Ant Colony Optimization - ACO) elvével heurisztikus, valószínűségi megoldást találni. Az ACO nagymértékben párhuzamosítható, ezért az NVIDIA CUDA keretrendszer segítségével videokártyán összehangolt munkaszálak ezreit futtatva hatékonyan voltam képes megoldani a VRP-t. A készült kódot C/C++ programozási nyelven készítettem el.

Az ACO nagymértékben párhuzamosítható, ezért jól kihasználhatóak az NVIDIA CUDA keretrendszer adta lehetőségek. A VRP ennek segítségével hatékonyan megvalósítható videokártyán, összehangolt munkaszálak ezreit futtatva.

\vfill
\selectenglish



%----------------------------------------------------------------------------
% Abstract in English
%----------------------------------------------------------------------------
\chapter*{Abstract}\addcontentsline{toc}{chapter}{Abstract}

We encounter Vehicle Routing Problems (VRP) in various aspects of life, such as analyzing the user base of a large social network, planning the layout design of a Printed Ciruit Board, or handling different logistics tasks like delivery problems. These are well-known NP-complete problems, therefore they are very costly to solve. The aim of this thesis is to use the principle of Ant Colony Optimization (ACO) method to find solutions heuristically, probabilistically. ACO is capable to be highy parallelized that I could utilize by running thousands of synchronized threads on a graphics card (GPU) using NVIDIA CUDA framework. The developed code was implemented in C/C++.


\vfill
\selectthesislanguage

\newcounter{romanPage}
\setcounter{romanPage}{\value{page}}
\stepcounter{romanPage}